\documentclass[primordial-cosmological-perturbations.tex]{subfiles}
\begin{document}

\section{Disformal Transformation}

\subsection{Perturbing the FLRW Metric}

Let us decompose the perturbed FLRW metric (with SVT decomposition) as
\begin{equation}
\label{eq:perturbedFLRW}
ds^{2}
= -(1+2\varphi)\,dt^{2}
+ 2a(\alpha_{,i}+\beta_{i})\,dt\,dx^{i}
+ a^{2}\Big[(1-2\psi)\delta_{ij}+\gamma_{ij}+2E_{,ij}+2F_{(i,j)}\Big]\,dx^{i}dx^{j},
\end{equation}
where $a=a(t)$ is the scale factor, $(\varphi,\alpha,\psi,E)$ are scalar perturbations, $\beta_{i},F_{i}$ are vector perturbations, and $\gamma_{ij}$ are tensor perturbations.

Under a disformal transformation, the line element becomes
\begin{equation}
d\hat{s}^{2}=\hat{g}_{\mu\nu}\,dx^{\mu}dx^{\nu}.
\end{equation}

The invertible disformal transformation with higher derivatives presented by Takahashi, Motohashi, and Minamitsuji (which we call the TMM transformation for convenience) is
\begin{equation}
\label{eq:TMM}
\hat{g}_{\mu\nu}
= A\,g_{\mu\nu}
+ B\,\phi_{;\mu}\phi_{;\nu}
+ C\big(\phi_{;\mu}X_{;\nu}+X_{;\mu}\phi_{;\nu}\big)
+ D\,X_{;\mu}X_{;\nu},
\end{equation}
where $A,B,C,D$ are functionals of $\phi,X,Y,Z$, and $X,Y,Z$ are defined as
\begin{equation}
\label{eq:kinetic-terms}
X\equiv -\frac{1}{2}g^{\mu\nu}\phi_{;\mu}\phi_{;\nu},
\qquad
Y\equiv g^{\mu\nu}\phi_{;\mu}X_{;\nu},
\qquad
Z\equiv g^{\mu\nu}X_{;\mu}X_{;\nu}.
\end{equation}

The TMM form can be simplified by a ``completing-the-square'' step. We have ``squared'' covariant-derivative terms of $\phi$, cross terms in the middle, and ``squared'' covariant-derivative terms of $X$. Recall the elementary identity
\begin{equation}
\label{eq:completing-square}
ax^{2}+2bxy+cy^{2}
= a\Big(x+\frac{b}{a}y\Big)^{2}+\Big(c-\frac{b^{2}}{a}\Big)y^{2}.
\end{equation}
Identifying $a=B$, $b=C$, and $c=D$, the last three terms in \eqref{eq:TMM} combine as
\begin{equation}
\label{eq:square-step}
B\Big(\phi_{;\mu}+\frac{C}{B}X_{;\mu}\Big)\Big(\phi_{;\nu}+\frac{C}{B}X_{;\nu}\Big)
+\Big(D-\frac{C^{2}}{B}\Big)X_{;\mu}X_{;\nu}.
\end{equation}
Hence, \eqref{eq:TMM} may be rewritten as
\begin{equation}
\label{eq:TMM-square}
\hat{g}_{\mu\nu}
= A\,g_{\mu\nu}
+ B\Big(\phi_{;\mu}+\frac{C}{B}X_{;\mu}\Big)\Big(\phi_{;\nu}+\frac{C}{B}X_{;\nu}\Big)
+ \Big(D-\frac{C^{2}}{B}\Big)X_{;\mu}X_{;\nu}.
\end{equation}

If we define
\begin{equation}
\label{eq:Phi-Delta-def}
\Phi_{\mu}\equiv \sqrt{B}\Big(\phi_{;\mu}+\frac{C}{B}X_{;\mu}\Big),
\qquad
\Delta\equiv D-\frac{C^{2}}{B},
\end{equation}
then the transformation reduces cleanly to
\begin{equation}
\label{eq:simplifiedTMM}
\boxed{\hat{g}_{\mu\nu}=A\,g_{\mu\nu}+\Phi_{\mu}\Phi_{\nu}+\Delta\,X_{;\mu}X_{;\nu}.}
\end{equation}

Under \eqref{eq:simplifiedTMM}, the transformed line element can be expanded as
\begin{equation}
\label{eq:dshat-expanded}
\begin{aligned}
d\hat{s}^{2}
&= \hat{g}_{\mu\nu}\,dx^{\mu}dx^{\nu} \\
&= \big(Ag_{00}+\Phi_{0}\Phi_{0}+\Delta\,X_{;0}X_{;0}\big)\,dt^{2} \\
&\quad + 2\big(Ag_{0i}+\Phi_{0}\Phi_{i}+\Delta\,X_{;0}X_{;i}\big)\,dt\,dx^{i} \\
&\quad + \big(Ag_{ij}+\Phi_{i}\Phi_{j}+\Delta\,X_{;i}X_{;j}\big)\,dx^{i}dx^{j}.
\end{aligned}
\end{equation}

Consequently, take the perturbed scalar field $\phi=\bar{\phi}+\delta\phi$. Then
\begin{equation}
\label{eq:X-expand}
\begin{aligned}
X
&= -\frac{1}{2}g^{\mu\nu}\phi_{;\mu}\phi_{;\nu} \\
&= -\frac{1}{2}g^{00}\dot{\phi}^{\,2} \\
&= -\frac{1}{2}\big(-(1-2\varphi)\big)\,(\dot{\bar{\phi}}+\delta\dot{\phi})^{2} \\
&= \Big(\frac{1}{2}-\varphi\Big)\Big(\dot{\bar{\phi}}^{2}+2\dot{\bar{\phi}}\,\delta\dot{\phi}+(\delta\dot{\phi})^{2}\Big) \\
&= \frac{1}{2}\dot{\bar{\phi}}^{2}+\dot{\bar{\phi}}\,\delta\dot{\phi}-\varphi\,\dot{\bar{\phi}}^{2},
\end{aligned}
\end{equation}
where $g^{00}$ comes from the perturbed metric, and we have dropped higher-order perturbations (e.g.\ $(\delta\dot{\phi})^{2}$).

It is natural to split $A, \Phi_{0}$, and $\Delta$ as 
\begin{equation}
A = \bar { A } + \delta A , \qquad \Phi _ { 0 } = \bar { \Phi } _ { 0 } + \delta \Phi _ { 0 } , \qquad \Delta = \bar { \Delta } + \delta \Delta ,
\label{eq:splits}
\end{equation}

Notice that from (\ref{eq:Phi-Delta-def}) we have 

\begin{equation}
\begin{aligned}
\Phi_{0} &= \sqrt { B } \Big ( \phi _ { ; 0 } + \cfrac { C } { B } X _ { ; 0} \Big  ) \\
         &= \sqrt{B} \Big ( \dot{\bar{\phi}} + \delta\dot{\phi} +\frac{C}{B}(\dot{\bar{X}}+\delta\dot{X}) \Big)
\end{aligned}
\label{eq:label}
\end{equation}

Where we see to have zeroth order perturbations given back the background variables. On the other hand for the spatial component, notice 

\begin{equation}
\begin{aligned}
\Phi_{i} &= \sqrt { B } \Big ( \phi _ { ; i } + \cfrac { C } { B } X _ { ; i} \Big  ) \\
         &= \sqrt { B } \Big (\cfrac { C } { B } (\bar{X_{,i}} + \delta X_{,i}) \Big  ) \\ 
         &= \sqrt { B } \Big (\cfrac { C } { B } \delta X_{,i} \Big )
\end{aligned}
\label{eq:label}
\end{equation}

Where again, the background term vanishes under a spatial covariant derivative since it is only dependent upon time. On the other hand, 
we are left with $\delta X_{,i}$ which is a first order perturbation. Hence, we choose not to "split" the spatial components of $\Phi_{\mu}$ 

\subsection{For the $dt^2$ terms}

From the perturbed metric given by (\ref{eq:perturbedFLRW}) we have $g_{00} = -(1=2\varphi)$  it follows that 
\begin{equation}
\begin{aligned}
Ag_{{00}} &= -(\bar{A} + \delta A)(1+2\varphi) \\ 
          &= -(\bar{A} + 2\varphi \bar{A} +\delta A + 2\varphi \delta A) \\
          &\approx -\bar{A} -\delta A - 2 \bar{A}\varphi
\end{aligned}
\label{eq:Ag_00}
\end{equation}

On the other hand
\begin{equation}
\Phi_{0}^2 = (\bar{\Phi_{0}} + \delta\Phi_{0})^2 = \bar{\Phi_{0}}^2 + 2\Phi_{o}
\label{eq:Phi0sq}
\end{equation}

Finally, 
\begin{equation}
\begin{aligned}
\Delta X_{;0}X_{;0} &= (\bar{\Delta} + \delta \Delta) (\dot{\bar{X}} + \delta X)^2 \\
                    &= (\bar{\Delta} + \delta \Delta)(\dot{\bar{X}}^2 + 2\bar{\dot{X}}\delta X + \overbrace{(\delta X)^2)}^\text{2nd order} \\ 
                    &\approx \bar{\Delta} \dot{\bar{X}}^2 + \delta \Delta \dot{\bar{X}}^2 + 2\bar{\Delta} \dot{\bar{X}} \delta X + \underbrace{2\delta \Delta \dot{\bar{X}}\delta X}_{\text{2nd order}} \\
                    &\approx \bar{\Delta} \dot{\bar{X}}^2 + \delta \Delta \dot{\bar{X}}^2 + 2\bar{\Delta} \dot{\bar{X}} \delta X  
\end{aligned}
\label{eq:DeltaX0sq}
\end{equation}


\subsection{For $dt dx^i$ terms}
Again, given by (\ref{eq:perturbedFLRW}) we find that $g_{0i} = a(\alpha_{,i} + \beta_{i})$. So for $Ag_{0i}$ we have 

\begin{equation}
\begin{aligned}
Ag_{0i} &= a(\bar{A} + \delta A)(\alpha_{,i} + \beta_{i})\\
        &=a(\bar{A} \alpha_{,i} + \bar{A}\beta_{i} + \delta A \alpha_{,i} + \delta A \beta_{i}) \\
        &\approx a \bar{A}(\alpha_{,i} + \beta_{i})
\end{aligned}
\label{eq:Ag_0i}
\end{equation}

For $\Phi_{0} \Phi_{i}$ note that we have already established that $\Phi_{i}$ is a already a first order perturbation. Then 

\begin{equation}
\begin{aligned}
\Phi_{0}\Phi_{i} &= (\bar{\Phi_{0}} + \delta \Phi_{0})(\Phi_{i}) \\ 
                 &= \bar{\Phi_{0}}\Phi_{i} + \delta \Phi_{0}\Phi_{i} \\ 
                 &\approx \bar{\Phi_{0}}\Phi_{i}
\end{aligned}
\label{eq:Phi0Phii}
\end{equation}

On the other hand, for $\Delta X_{{;0}} X_{{;i}}$, note that $X_{{;i}} = \bar{X}_{;i} + \delta X_{;i} = \delta X_{;i}$ since background 
quantities are only time dependent. So,
\begin{equation}
\begin{aligned}
\Delta X_{{;0}} X_{{;i}} &= (\bar{\Delta} + \delta \Delta)(\bar{X}_{;0} + \delta X_{;0}) \delta X_{;i} \\ 
                         &= (\bar{\Delta} \dot{\bar{X}} + \bar{\Delta}\delta\dot{X} + \delta \Delta \dot{\bar{X}} + \delta \Delta \delta \dot{X}) \delta X_{,i} \\
                         &\approx (\bar{\Delta} \dot{\bar{X}} + \bar{\Delta}\delta\dot{X} + \delta \Delta \dot{\bar{X}}) \delta X_{,i} \\
                         &\approx \bar{\Delta}\dot{\bar{X}}\delta X_{,i}
\end{aligned}
\label{eq:label}
\end{equation}

\subsection{For $dx^i dx^j$ terms}

\end{document}

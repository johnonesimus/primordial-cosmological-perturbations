\documentclass[primordial-cosmological-perturbations.tex]{subfiles}
\begin{document}

\section{The TMM Transformation} 
Let us decompose the perturbed FLRW metric (with SVT decomposition) as
\begin{equation}
\label{eq:perturbedFLRW}
ds^{2}
= -(1+2\varphi)\,dt^{2}
+ 2a(\alpha_{,i}+\beta_{i})\,dt\,dx^{i}
+ a^{2}\Big[(1-2\psi)\delta_{ij}+\gamma_{ij}+2E_{,ij}+2F_{(i,j)}\Big]\,dx^{i}dx^{j},
\end{equation}
where $a=a(t)$ is the scale factor, $(\varphi,\alpha,\psi,E)$ are scalar perturbations, $\beta_{i},F_{i}$ are vector perturbations, and $\gamma_{ij}$ are tensor perturbations.
Note that throughout this transformation a partial derivative is denoted by a comma subscript "$,$". 
Meanwhile, a covariant derivative is denoted by a semi-colon subscript "$;$".

Under a disformal transformation, the line element becomes
\begin{equation}
d\hat{s}^{2}=\hat{g}_{\mu\nu}\,dx^{\mu}dx^{\nu}.
\end{equation}

The invertible disformal transformation with higher derivatives presented by Takahashi, Motohashi, and Minamitsuji (which we call the TMM transformation for convenience) is
\begin{equation}
\label{eq:TMM}
\hat{g}_{\mu\nu}
= A\,g_{\mu\nu}
+ B\,\phi_{;\mu}\phi_{;\nu}
+ C\big(\phi_{;\mu}X_{;\nu}+X_{;\mu}\phi_{;\nu}\big)
+ D\,X_{;\mu}X_{;\nu},
\end{equation}
where $A,B,C,D$ are functionals of $\phi,X,Y,Z$, and $X,Y,Z$ are defined as
\begin{equation}
\label{eq:kinetic-terms}
X\equiv -\frac{1}{2}g^{\mu\nu}\phi_{;\mu}\phi_{;\nu},
\qquad
Y\equiv g^{\mu\nu}\phi_{;\mu}X_{;\nu},
\qquad
Z\equiv g^{\mu\nu}X_{;\mu}X_{;\nu}.
\end{equation}


\section{Expanding $d\hat{s}^2$ term-by-term}

Under \eqref{eq:TMM}, the transformed line element can be expanded as
\begin{equation}
\label{eq:dshat-expanded}
\begin{aligned}
d\hat{s}^{2}
&= \hat{g}_{\mu\nu}\,dx^{\mu}dx^{\nu} \\
&= \big(Ag_{00} + B \phi_{;0}^2  + C(2\phi_{;0}X_{;0})  +DX_{;0}^2    \big) dt^2 \\ 
&= 2\big(Ag_{0i} + B \phi_{;0}\phi_{;i}  + C(\phi_{;0}X_{;i}+X_{;0}\phi_{;i} )  +DX_{;0}X_{;i}  \big)dt dx^i\\
&= \big(Ag_{ij} + B \phi_{;i}\phi_{;j}  + C(\phi_{;i}X_{;j}+X_{;i}\phi_{;j} )  +DX_{;i}X_{;j}  \big)dx^i dx^j
\end{aligned}
\end{equation}




\section{Background-perturbation split}
We perturb the scalar field as 
\begin{equation}
\phi(t, \mathbf{x}) = \bar{\phi(t)} + \delta \phi(t,\mathbf{x})
\end{equation}

Since this is a scalar field the covariant derivative reduces to partial derivatives, it follows that
\begin{equation}
\phi _ { ; 0 } = \partial _ { 0 } \phi = \dot { { \bar { \phi } } } + \delta \dot { \phi } , \qquad \phi _ { ; i } = \partial _ { i } \phi = \delta\phi_{,i} ,
\label{eq:phi_derivs}
\end{equation}

where we used $\bar{\phi}_{,i} = 0$ by background homogeneity.  

Likewise for $X$ we have
\begin{equation}
X(x,\mathbf{x}) = \bar{X} + \delta X
\end{equation} 

Consequently,

\begin{equation}
X_{;0} = \dot{\bar{X}} +\delta \dot{X}, \qquad X_{;i} = \delta X_{,i}
\label{eq:xderivs}
\end{equation}

It should also make sense to split the coefficients since they are functionals of $(\phi, X,Y,Z)$. So we have 

\begin{equation}
A = \bar { A } + \delta A , \qquad B = \bar { B } + \delta B , \qquad C = \bar { C } + \delta C , \qquad D = \bar { D } + \delta D ,
\label{eq:ABCDsplit}
\end{equation}

\subsection{Linear Expression for $X$ }
\begin{equation}
\label{eq:X-expand}
\begin{aligned}
X
&= -\frac{1}{2}g^{\mu\nu}\phi_{;\mu}\phi_{;\nu} \\
&= -\frac{1}{2}g^{00}\dot{\phi}^{\,2} \\
&= -\frac{1}{2}\big(-(1-2\varphi)\big)\,(\dot{\bar{\phi}}+\delta\dot{\phi})^{2} \\
&= \Big(\frac{1}{2}-\varphi\Big)\Big(\dot{\bar{\phi}}^{2}+2\dot{\bar{\phi}}\,\delta\dot{\phi}+(\delta\dot{\phi})^{2}\Big) \\
&= \frac{1}{2}\dot{\bar{\phi}}^{2}+\dot{\bar{\phi}}\,\delta\dot{\phi}-\varphi\,\dot{\bar{\phi}}^{2},
\end{aligned}
\end{equation}
where $g^{00} =-(1-2\varphi)$ follows from inverting then expanding $g_{00}$ at linear order.  We have also dropped higher-order perturbations (e.g.\ $(\delta\dot{\phi})^{2}$).

\section{Linear Expansion of the Transformed Metric}

We are now ready to expand (\ref{eq:dshat-expanded}), using the splits and their derivatives given by \cref{eq:phi_derivs,eq:xderivs,eq:ABCDsplit}, while 
discarding higher-order perturbations (in other words, $\mathcal{O}(\epsilon^2)$ terms)

\subsection{The $dt^2$ terms}
We wish to expand 
\begin{equation}
A g _ { 0 0 } + B \phi _ { ; 0 } ^ { 2 } + C ( 2 \phi _ { ; 0 } X _ { ; 0 } ) + D X _ { ; 0 } ^ { 2 }
\end{equation}

\subsubsection*{For $Ag_{00}$} 
From \cref{eq:perturbedFLRW}, we have $g_{00} = -(1+2\varphi)$ 

Hence,
\begin{equation}
\begin{aligned}
Ag_{{00}} &= -(\bar{A} + \delta A)(1+2\varphi) \\ 
          &= -(\bar{A} + 2\varphi \bar{A} +\delta A + 2\varphi \delta A) \\
          &\approx -\bar{A} -\delta A - 2 \bar{A}\varphi
\end{aligned}
\label{eq:Ag_00}
\end{equation}

because $ 2\varphi \delta A$ is $\mathcal{O}(\epsilon^2)$. 

Starting from here we will be introducing a new notation for cases where $\mathcal{O}(\epsilon^n), n\geq 2$. For a particular term 
that inherits higher-ordered perturbations we will be cancelling them as follows: $\cancelto{}{\text{term}}$.

\subsubsection*{For $B \phi_{;0}^2$ }

Using \cref{eq:ABCDsplit,eq:phi_derivs} we have 

\begin{equation}
\begin{aligned}
B \phi_{;0}^2 &=(\bar{B} + \delta B)(\dot{\bar{\phi}} + \delta\dot{\phi})^2 \\ 
              &=(\bar{B} + \delta B)(\dot{\bar{\phi}}^2 + 2\dot{\bar{\phi}}\delta\dot{\phi} + \cancelto{}{\delta \dot{\phi}^2}) \\ 
              &\approx \bar{B}\dot{\bar{\phi}}^2 + 2\bar{B}\dot{\bar{\phi}}\delta\dot{\phi} + \delta B \dot{\bar{\phi}}^2 + \cancelto{}{2 \dot{\phi} \delta B} \delta \dot{\phi} \\ 
              &\approx \bar{B}\dot{\bar{\phi}}^2 + 2\bar{B}\dot{\bar{\phi}}\delta\dot{\phi} + \delta B \dot{\bar{\phi}}^2 
\end{aligned}
\label{eq:Bphi}
\end{equation}

\subsubsection*{For the cross-term $C \phi_{;0} X_{;0}$ we have }

\begin{equation}
\begin{aligned}
C \phi_{;0} X_{;0} &= (\bar{C} + \delta C)(\dot{\bar{\phi}} + \delta\dot{\phi})(\dot{\bar{X}} + \delta\dot{X}) \\ 
                   &= (\bar{C} + \delta C)(\dot{\bar{\phi}}\dot{\bar{X}} +\dot{\bar{\phi}} \delta\dot{X} + \delta\dot{\phi} \dot{\bar{X}} +  \cancelto{}{\delta\dot{\phi}\delta \dot{X}}) \\ 
                   &\approx \bar{C}\dot{\bar{\phi}}\dot{\bar{X}} + \bar{C}\dot{\bar{\phi}} \delta\dot{X} + \bar{C}\delta\dot{\phi} \dot{\bar{X}} + \delta C\dot{\bar{\phi}}\dot{\bar{X}}
                    + \cancelto{}{\delta C\dot{\bar{\phi}} \delta\dot{X}} + \cancelto{}{\delta C\delta\dot{\phi} \dot{\bar{X}}} \\ 
                   &\approx \bar{C}\dot{\bar{\phi}}\dot{\bar{X}} + \bar{C}\dot{\bar{\phi}} \delta\dot{X} + \bar{C}\delta\dot{\phi} \dot{\bar{X}} + \delta C\dot{\bar{\phi}}\dot{\bar{X}} 
\end{aligned}
\label{eq:Cphix}
\end{equation}

\subsubsection*{For $D X _ { ; 0 } ^ { 2 }$ }
Finally we have, 

\begin{equation}
\begin{aligned}
D X _ { ; 0 } ^ { 2 } &=(\bar{D} + \delta D)(\dot{\bar{X}} + \delta\dot{X})^2 \\ 
                      &=(\bar{D} + \delta D)(\dot{\bar{X}}^2 + 2\dot{\bar{X}}\delta\dot{X} + \cancelto{}{(\delta\dot{X})^2}) \\ 
                      &\approx \bar{D}\dot{\bar{X}}^2 + 2\bar{D}\dot{\bar{X}}\delta\dot{X} + \delta D\dot{\bar{X}}^2 + \cancelto{}{2\delta D \dot{\bar{X}}\delta\dot{X}} \\ 
                      &\approx \bar{D}\dot{\bar{X}}^2 + 2\bar{D}\dot{\bar{X}}\delta\dot{X} + \delta D\dot{\bar{X}}^2 
\end{aligned}
\label{eq:DXsqr}
\end{equation}

Putting \cref{eq:Ag_00,eq:Bphi,eq:Cphix,eq:DXsqr}, therefore $\hat{g_{00}}$ in linear-order is

\begin{equation}
\boxed{
\begin{aligned}
\hat{g}_{00} \approx &-\bar{A} -\delta A - 2 \bar{A}\varphi \\
&+ \bar{B}\dot{\bar{\phi}}^2 + 2\bar{B}\dot{\bar{\phi}}\delta\dot{\phi} + \delta B \dot{\bar{\phi}}^2 \\ 
&+ 2(\bar{C}\dot{\bar{\phi}}\dot{\bar{X}} + \bar{C}\dot{\bar{\phi}} \delta\dot{X} + \bar{C}\delta\dot{\phi} \dot{\bar{X}} + \delta C\dot{\bar{\phi}}\dot{\bar{X}}) \\ 
&+ \bar{D}\dot{\bar{X}}^2 + 2\bar{D}\dot{\bar{X}}\delta\dot{X} + \delta D\dot{\bar{X}}^2 
\end{aligned}}
\label{eq:ghat00}
\end{equation}


\section{For the $dt dx^i$  terms}
We wish to expand $Ag_{0i} + B \phi_{;0}\phi_{;i}  + C(\phi_{;0}X_{;i}+X_{;0}\phi_{;i} )  +DX_{;0}X_{;i}$. 

\subsubsection*{For $Ag_{0i}$ }
\begin{equation}
\begin{aligned}
Ag_{0i} &= a(\bar{A} + \delta A)(\alpha_{,i} + \beta_{i})\\
        &=a(\bar{A} \alpha_{,i} + \bar{A}\beta_{i} + \cancelto{}{\delta A \alpha_{,i}} + \cancelto{}{\delta A \beta_{i}}) \\
        &\approx a \bar{A}(\alpha_{,i} + \beta_{i})
\end{aligned}
\label{eq:Ag_0i}
\end{equation}


\subsubsection*{For $ B \phi_{;0}\phi_{;i} $}

\begin{equation}
\begin{aligned}
 B \phi_{;0}\phi_{;i} &= (\bar{B} + \delta B)(\dot{\bar{\phi}} + \delta \dot{\phi})(\delta \phi_{,i}) \\ 
                      &= (\bar{B} + \delta B)(\delta \phi_{,i}\dot{\bar{\phi}} + \cancelto{}{\delta \phi_{,i}\delta \dot{\phi}}) \\ 
                      &\approx \bar{B}\delta \phi_{,i}\dot{\bar{\phi}} + \cancelto{}{\delta B \delta \phi_{,i}\dot{\bar{\phi}} } \\ 
                      &\approx \bar{B} \dot{\bar{\phi}}\delta\phi_{,i}
\end{aligned}
\label{eq:Bphi_0i}
\end{equation}


\subsubsection*{For $C ( \phi _ { ; 0 } X _ { ; i } + X _ { ; 0 } \phi _ { ; i } )$ } 

\begin{equation}
\begin{aligned}
C (\phi _{ ; 0 }X _{ ; i } + X _ { ; 0 } \phi _ { ; i }) &= ( \bar { C } + \delta C ) \Big [ ( \dot { \bar { \phi } } + \delta \dot { \phi } ) (\delta X_{,i} ) + ( \dot { \bar { X } } + \delta \dot { X } ) (\delta \phi_{,i} ) \Big ] \\ 
                                                         &= (\bar{C} + \delta C) \Big[ \delta X_{,i}\dot{\bar { \phi } } + \cancelto{}{\delta X_{,i}\delta \dot { \phi }} + \delta \phi_{,i}\dot { \bar { X } } + \cancelto{}{\delta \phi_{,i}\delta \dot { X } }   \Big] \\ 
                                                         &\approx \bar{C}\delta X_{,i}\dot{\bar { \phi } } + \bar{C}\delta \phi_{,i}\dot { \bar { X } } + \cancelto{}{\delta C \delta X_{,i}\dot{\bar { \phi }} } + \cancelto{}{\delta C\delta \phi_{,i}\dot { \bar { X } }} \\ 
                                                         &\approx \bar{C}(\delta X_{,i}\dot{\bar { \phi } } + \delta \phi_{,i}\dot { \bar { X } } )
\end{aligned}
\label{eq:Cphi0xix0phii}
\end{equation}

\subsubsection*{For $D X_{;0}X_{;i}$}

\begin{equation}
\begin{aligned}
D X_{;0}X_{;i} &= (\bar{D} + \delta D) (\dot{\bar{X}} + \delta \dot{X})\delta X_{,i} \\ 
               &= (\bar{D} + \delta D) (\delta X_{,i}\dot{\bar{X}} + \cancelto{}{\delta X_{,i}\delta \dot{X}}) \\ 
               &\approx \bar{D}\delta X_{,i}\dot{\bar{X}} + \cancelto{}{\delta D\delta X_{,i}\dot{\bar{X}}} \\ 
               &\approx \bar{D}\dot{\bar{X}}\delta X_{,i}
\end{aligned}
\label{eq:Dx0xi}
\end{equation}


Finally, putting together \cref{eq:Ag_0i,eq:Bphi_0i,eq:Cphi0xix0phii,eq:Dx0xi} $\hat{g}_0i$ in linear order is given by 
\begin{equation}
\boxed{
\hat{g}_{0i} \approx a \bar{A}(\alpha_{,i} + \beta_{i}) + \bar{B} \dot{\bar{\phi}}\delta\phi_{,i} + \bar{C}(\delta X_{,i}\dot{\bar { \phi } } + \delta \phi_{,i}\dot { \bar { X } }) + \bar{D}\dot{\bar{X}}\delta X_{,i}
}
\label{eq:ghat0i}
\end{equation} 




\subsection{For $dx^i dx^j$ terms}

We wish to expand $Ag_{ij} + B \phi_{;i}\phi_{;j}  + C(\phi_{;i}X_{;j}+X_{;i}\phi_{;j} )  +DX_{;i}X_{;j}$.
Now before we proceed, we have established from \cref{eq:phi_derivs,eq:xderivs} that the perturbations with derivatives 
with respect to spatial components are already first-order. Hence this reduces to 

\begin{equation}
\begin{aligned}
Ag_{ij} + \cancelto{}{B \phi_{;i}\phi_{;j}}  + \cancelto{}{C(\phi_{;i}X_{;j}+X_{;i}\phi_{;j} ) } +\cancelto{}{DX_{;i}X_{;j}}\\ 
\end{aligned}
\end{equation}


A huge shout-out to cosmological perturbations for making the math bearable. From \cref{eq:perturbedFLRW} 
we have $g_{ij} = a^{2}\Big[(1-2\psi)\delta_{ij}+\gamma_{ij}+2E_{,ij}+2F_{(i,j)}\Big]$; then

\begin{equation}
\begin{aligned}
Ag_{ij} =&\,  a^2 (\bar{A} + \delta A) \Big[(1-2\psi)\delta_{ij}+\gamma_{ij}+2E_{,ij}+2F_{(i,j)}\Big] \\ 
        =&\, a^2 \bar{A}\Big[(1-2\psi)\delta_{ij}+\gamma_{ij}+2E_{,ij}+2F_{(i,j)}\Big] \\ 
           &+  \delta A \Big[(1-2\psi)\delta_{ij}+\gamma_{ij}+2E_{,ij}+2F_{(i,j)}\Big] \\ 
        \approx&\, a^2 \bar{A}\Big[(1-2\psi)\delta_{ij}+\gamma_{ij}+2E_{,ij}+2F_{(i,j)}\Big] + a^2\delta A \delta_{ij} \\ 
        \approx&\, a^2 \bar{A} [(1-2\psi + \delta A / \bar{A})\delta_{ij} +\gamma_{ij}+2E_{,ij}+2F_{(i,j)} ]
\end{aligned}
\label{eq:label}
\end{equation}

Then it follows that 
\begin{equation}
\boxed{
\hat{g}_{ij} \approx a^2 \bar{A} [(1-2\psi + \delta A / \bar{A})\delta_{ij} +\gamma_{ij}+2E_{,ij}+2F_{(i,j)} ]
}
\label{eq:ghatij}
\end{equation}

\section{Finally $d\hat{s}^2$ }
And so with
\begin{equation}
d \hat { s } ^ { 2 } = \hat { g } _ { 0 0 } \, d t ^ { 2 } + 2 \hat { g } _ { 0 i } \, d t \, d x ^ { i } + \hat { g } _ { i j } \, d x ^ { i } d x ^ { j }
\end{equation} 

and collecting our results given by \cref{eq:ghat00,eq:ghat0i,eq:ghatij} we finally have 


\begin{equation}
\begin{aligned}
d\widehat{s}^2=\;&
\Big[
-\bar{A}-\delta A-2\bar{A}\varphi
+\bar{B}\dot{\bar{\phi}}^{2}+2\bar{B}\dot{\bar{\phi}}\,\delta\dot{\phi}+\delta B\,\dot{\bar{\phi}}^{2}
\\
&\quad
+2\Big(\bar{C}\dot{\bar{\phi}}\dot{\bar{X}}
+\bar{C}\dot{\bar{\phi}}\,\delta\dot{X}
+\bar{C}\delta\dot{\phi}\,\dot{\bar{X}}
+\delta C\,\dot{\bar{\phi}}\dot{\bar{X}}\Big)
\\
&\quad
+\bar{D}\dot{\bar{X}}^{2}+2\bar{D}\dot{\bar{X}}\,\delta\dot{X}+\delta D\,\dot{\bar{X}}^{2}
\Big]\,dt^{2}
\\
&\;+\;2\Big[
a\bar{A}(\alpha_{,i}+\beta_{i})
+\bar{B}\dot{\bar{\phi}}\,\delta\phi_{,i}
+\bar{C}\big(\dot{\bar{\phi}}\,\delta X_{,i}+\dot{\bar{X}}\,\delta\phi_{,i}\big)
+\bar{D}\dot{\bar{X}}\,\delta X_{,i}
\Big]\,dt\,dx^{i}
\\
&\;+\;
a^{2}\bar A \Big[\Big(1-2\psi+\frac{\delta A}{\bar A}\Big)\delta_{ij}
+\gamma_{ij}+2E_{,ij}+2F_{(i,j)}\Big]\,dx^{i}dx^{j}.
\end{aligned}
\label{eq:dshat-final}
\end{equation}

Cleaning it up a bit to match Ref. \cite{alineaTransformationPrimordialCosmological2021}, then

\begin{equation}
\begin{aligned}
d\widehat{s}^{2}=
\Big[
&-\bar{A}-\delta A-2\bar{A}\varphi
+\bar{B}\dot{\bar{\phi}}^{2}+2\bar{B}\dot{\bar{\phi}}\,\delta\dot{\phi}+\delta B\,\dot{\bar{\phi}}^{2}
\\
\quad
&+2\Big(\bar{C}\dot{\bar{\phi}}\dot{\bar{X}}
+\bar{C}\dot{\bar{\phi}}\,\delta\dot{X}
+\bar{C}\delta\dot{\phi}\,\dot{\bar{X}}
+\delta C\,\dot{\bar{\phi}}\dot{\bar{X}}\Big)
\\
\quad
&+\bar{D}\dot{\bar{X}}^{2}+2\bar{D}\dot{\bar{X}}\,\delta\dot{X}+\delta D\,\dot{\bar{X}}^{2}
\Big]\,dt^{2}
\\
&+2a\bar{A}^{\frac{1}{2}}\Big[
\bar{A}^{\frac{1}{2}}(\alpha_{,i}+\beta_{i})
+\big[\bar{B}\dot{\bar{\phi}}\,\delta\phi_{,i}
+\bar{C}\big(\dot{\bar{\phi}}\,\delta X_{,i}+\dot{\bar{X}}\,\delta\phi_{,i}\big)
+\bar{D}\dot{\bar{X}}\,\delta X_{,i}\big] \big/a \bar{A}^{\frac{1}{2}}
\Big]\,dt\,dx^{i}
\\
&+
a^{2}\bar A \Big[\Big(1-2\psi+\frac{\delta A}{\bar A}\Big)\delta_{ij}
+\gamma_{ij}+2E_{,ij}+2F_{(i,j)}\Big]\,dx^{i}dx^{j}.
\end{aligned}
\label{eq:dshat-final2}
\end{equation}

If we switch off all perturbations, then the hatted metric becomes 

\begin{equation}
\begin{aligned}
d\widehat{s} = - \Big[\bar{A} + \bar{B}\dot{\bar{\phi}}^{2} + 2\bar{C}\dot{\bar{\phi}}\dot{\bar{X}} + \bar{D}\dot{\bar{X}}^2\Big]dt^2 + a^2\bar{A}\delta_{ij}dx^idx^j
\end{aligned}
\label{eq:dshat-noperturb}
\end{equation}

For brevity let us define 
\begin{equation}
\mathcal{N} \equiv \bar{A} + \bar{B}\dot{\bar{\phi}}^{2} + 2\bar{C}\dot{\bar{\phi}}\dot{\bar{X}} + \bar{D}\dot{\bar{X}}^2
\label{eq:Ncal}
\end{equation}

And so the hatted metric cleanly reduces to 
\begin{equation}
d\widehat{s} = - \mathcal{N}dt^2 + a^2\bar{A}\delta_{ij}dx^idx^j
\label{eq:dshat-noperturb-clean}
\end{equation}

Now, \cref{eq:dshat-noperturb-clean}, following the form of the background FLRW metric, 
is suggestive of coordinate time and scale factor scaling. Following the idea for pure conformal transformation,
we may then define the hatted time coordinate and scale factor as 

\begin{equation}
\mathrm { d } \widehat { t } ^ { 2 } \equiv \mathcal{N} \mathrm { d } t ^ { 2 } \quad \text{ and } \quad \widehat { a } \equiv a \bar { A } ^ { \frac 1 2 }
\label{eq:hatted-time-scale}
\end{equation}

We may also simply write the perturbed FLRW metric as 
\begin{equation}
d\widehat{s}^2 =-(1+2\widehat{\varphi})d\widehat{t}^2 + 2\widehat{a}(\widehat{\alpha}_,i + \widehat{\beta}_i)d\widehat{t} dx^i 
+\widehat{a}^2 \Big[(1-2\widehat{\psi})\delta_{ij} + \widehat{\gamma}_{ij} + 2\widehat{E}_{,ij}+2\widehat{F}_{(i,j)}\Big]dx^i dx^j
\label{eq:label}
\end{equation}

Also, again recall that 

\begin{equation}
d\widehat{s}^2 = \widehat{g}_{00}dt^2 + 2\widehat{g}_{0i}dt dx^i + \widehat{g}_{ij}dx^idx^j
\label{eq:label}
\end{equation}

Then
\begin{equation}
-(1+2\widehat{\varphi})d \widehat{t}^2 = \underbrace{-(1+ 2\widehat{\varphi})\mathcal{N}}_{\widehat{g}_{00}}dt^2
\label{eq:label}
\end{equation}

Again recall that we've recovered $\widehat{g}_{00}$
for linear order perturbations given by \cref{eq:ghat00}. We can define 
the first order perturbations as 

\begin{equation}
\begin{aligned}
\widetilde{\mathcal{N}} \equiv &-\delta A -2\bar{A}\varphi + 2\bar{B}\dot{\bar{\phi}}\delta\dot{\phi} + \delta B \dot{\bar{\phi}}^2  \\
&+ 2(\bar{C}\dot{\bar{\phi}} \delta\dot{X} + \bar{C}\delta\dot{\phi} \dot{\bar{X}} + \delta C\dot{\bar{\phi}}\dot{\bar{X}}) \\ 
&+ 2\bar{D}\dot{\bar{X}}\delta\dot{X} + \delta D\dot{\bar{X}}^2 
\end{aligned}
\label{eq:label}
\end{equation}

Thus we may write \cref{eq:ghat00} as 
\begin{equation}
\widehat{g}_{00} = -\mathcal{N} - (2\bar{A}\varphi + \widetilde{\mathcal{N}})
\label{eq:label}
\end{equation}

So we have



\begin{equation}
 -\mathcal{N} - (2\bar{A}\varphi + \delta \mathcal{N}) = -(1+ 2\widehat{\varphi})\mathcal{N} \\ 
\label{eq:label}
\end{equation}

Finally,
\begin{equation}
\boxed{
\widehat{\varphi} = \frac{\bar{A} \varphi + \widetilde{\mathcal{N}} / 2}{\mathcal{N}}
}
\label{eq:00comp}
\end{equation}


On the other hand, for the $0i-$components we have 
\begin{equation}
\begin{aligned}
2\widehat a\big(\widehat\alpha_{,i}+\widehat\beta_i\big)
&=
\frac{2a\bar A^{1/2}}{\mathcal N^{\frac{1}{2}}}
\left\{
\bar A^{1/2}\big(\alpha_{,i}+\beta_i\big)
+\frac{
\bar B\,\dot{\bar\phi}\,\delta\phi_{,i}
+\bar C\!\left(\dot{\bar\phi}\,\delta X_{,i}+\dot{\bar X}\,\delta\phi_{,i}\right)
+\bar D\,\dot{\bar X}\,\delta X_{,i}}{a\,\bar A^{1/2}}
\right\} \\ 
&=
\frac{2\widehat{a}}{\mathcal N^{\frac{1}{2}}}
\left\{
\bar A^{1/2}\big(\alpha_{,i}+\beta_i\big)
+\frac{
\bar B\,\dot{\bar\phi}\,\delta\phi_{,i}
+\bar C\!\left(\dot{\bar\phi}\,\delta X_{,i}+\dot{\bar X}\,\delta\phi_{,i}\right)
+\bar D\,\dot{\bar X}\,\delta X_{,i}}{\widehat{a}}
\right\} 
\end{aligned}
\label{eq:0icomp}
\end{equation}

Owing to the Helmholtz decomposition, we know that any spatial vector field $V_{i}(\mathbf{x})$ can be uniquely 
written as $V_{i} + V_{,i} + V_{i}$ with $\partial^iV_{i}=0$ and $V_{,i}$ is the scalar mode. Clearly, the terms with a partial derivative 
are scalar and must contribute to the scalar shift $\widehat{\alpha}_{i}$. On the other hand the terms with no derivatives must contribute 
to the vector shift $\widehat{\beta}_i$. Matching the terms with their respective shifts therefore gives used

\begin{equation}
\begin{aligned}
\widehat{\alpha_{i}} &= \frac{\bar{A}^{\frac{1}{2}}}{\mathcal{N}^{\frac{1}{2}}}\alpha_{,i} + \frac{\bar B\,\dot{\bar\phi}\,\delta\phi_{,i}
+\bar C\!\left(\dot{\bar\phi}\,\delta X_{,i}+\dot{\bar X}\,\delta\phi_{,i}\right)
+\bar D\,\dot{\bar X}\,\delta X_{,i}}{\widehat{a}} \\
\widehat{\beta_{i}} &= \frac{\bar{A}^{\frac{1}{2}}}{\mathcal{N}^{\frac{1}{2}}}\beta_{i}
\end{aligned}
\label{eq:label}
\end{equation}

Finally, for the $ij$-components, by using similar (but this time a lot more simpler algebra) steps, we find 
\begin{equation}
\begin{aligned}
\widehat{\psi} & = \psi - \frac{\delta A}{2 \bar{A}},\\
\widehat{\gamma}_{ij} &= \gamma_{ij}, \\
\widehat{E} &= E, \\ 
\widehat{F} &= F
\end{aligned}
\label{eq:ijcomp}
\end{equation} 

Let us now move to a gauge-invariant scalar curvature perturbation which we define as 

\begin{equation}
\mathcal{R}_c \equiv \psi - \frac{H}{\dot{\bar{\phi}}}\delta \phi
 \label{eq:label}
\end{equation}

Here, $H$ is the Hubble parameter defined as $H \equiv \dot{a} / a$. It follows then 
that the disformally transformed perturbation is given by

\begin{equation}
\widehat{\mathcal{R}}_c = -\widehat{\psi} - \frac{\widehat{H}}{d\bar{\phi}/d\widehat{t}}\delta \phi
\label{eq:Rhat}
\end{equation}
        
We know that 
\begin{equation}
\widehat{H} = \frac{d \widehat{a}/d\widehat{t}}{\widehat{a}}
\label{eq:Hhat}
\end{equation}

Also, by the chain rule, while also taking note that $d\widehat{t}^2 = \mathcal{N} dt^2$ and $\widehat{a} = a \bar{A}^\frac{1}{2}$ we have 

\begin{equation}
\begin{aligned}
\frac{d\widehat{a}}{d\widehat{t}} &= \frac{d \widehat{a}}{dt} \frac{dt}{d\widehat{t}} \\ 
                                  &= \frac{d\widehat{a}}{dt} \mathcal{N}^{-\frac{1}{2}} \\
                                  &= \left( \dot{a}\bar{A}^{-\frac{1}{2}} + \frac{a \dot{\bar{A}}}{2 \bar{A}^{\frac{1}{2}}} \right) \mathcal{N}^{-\frac{1}{2}}
\end{aligned}
\label{eq:label}
\end{equation}

Then \cref{eq:Hhat} becomes 

\begin{equation}
\begin{aligned}
\widehat{H} &= \frac{1}{\widehat{a}} \frac{d\widehat{a}}{d\widehat{t}} \\
            &= \frac{1}{\widehat{a}} \left( \dot{a}\bar{A}^{\frac{1}{2}} + \frac{a \dot{\bar{A}}}{2 \bar{A}^{\frac{1}{2}}} \right) \mathcal{N}^{-\frac{1}{2}} \\ 
            &= \left( \frac{\dot{a}}{a} + \frac{\dot{\bar{A}}}{2\bar{A}}  \right)\mathcal{N}^{- \frac{1}{2}} \\ 
            &= \left( H + \frac{\dot{\bar{A}}}{2\bar{A}}  \right)\mathcal{N}^{- \frac{1}{2}}
\end{aligned}
\label{eq:label}
\end{equation}

By virtue of \cref{eq:ijcomp}, \cref{eq:Rhat} becomes 

\begin{equation}
\begin{aligned}
\widehat{R} &= -\psi + \frac{\delta A}{2\bar{A}} - (H + \frac{\dot{\bar{A}}}{2\bar{A}}) \frac{\delta \phi}{d \bar{\phi} / d \widehat{t}} \mathcal{N}^{-\frac{1}{2}} \\ 
            &= -\psi + \frac{\delta A}{2\bar{A}} - (H + \frac{\dot{\bar{A}}}{2\bar{A}}) \frac{\delta \phi}{\dot{\bar{\phi}} \mathcal{N}^{\frac{1}{2}}} \mathcal{N}^{-\frac{1}{2}} \\ 
            &= \underbrace{-\psi - H \frac{\delta \phi}{\dot{\bar{\phi}}}}_{\mathcal{R}_c} + \frac{\delta A}{2 \bar{A}} - \frac{\dot{\bar{A}}}{2 \bar{A}} \frac{\delta \phi}{\dot{\bar{\phi}}}
\end{aligned}
\label{eq:label}
\end{equation}

\begin{equation}
\boxed{
\widehat{\mathcal{R}_c}  = \mathcal{R}_c + \frac{\delta A}{2 \bar{A}} - \frac{\dot{\bar{A}}}{2 \bar{A}} \frac{\delta \phi}{\dot{\bar{\phi}}}
}
\label{eq:Rcalhat}
\end{equation}

\subsection*{Correction Terms in $\widehat{\mathcal{R}}_c$ }
We wish to explore the variation of the conformal factor given by $\delta A$. Note that $A$ is a functional of 
$(\phi, X,Y,Z)$, by the chain rule this spells out as

\begin{equation}
\frac { \delta A } { \bar { A } } = \frac { \bar { A } _ { , \bar { \phi } } } { \bar { A } } \, \delta \phi + \frac { \bar { A } _ { , \bar { X } } } { \bar { A } } \, \delta X + \frac { \bar { A } _ { , \bar { Y } } } { \bar { A } } \, \delta Y + \frac { \bar { A } _ { , \bar { Z } } } { \bar { A } } \, \delta Z .\label{eq:label}
\end{equation}


Also, the background term $\bar{A}$ is also a functional of $(\bar{\phi}, \bar{X}, \bar{Y}, \bar{Z})$, again by the chain rule 

\begin{equation}
\frac{\dot{\bar{A}}}{\bar{A}} = \frac{\bar{A}_{,\bar{\phi}}}{\bar{A}} \dot{\bar{\phi}} + \frac{\bar{A}_{,\bar{X}}}{\bar{A}} \dot{\bar{X}}
+  \frac{\bar{A}_{,\bar{Y}}}{\bar{A}} \dot{\bar{Y}} +  \frac{\bar{A}_{,\bar{Z}}}{\bar{A}} \dot{\bar{Z}}
\label{eq:Abarphibar}
\end{equation}

Isolating $\bar{A}_{,\bar{\phi}}/ \bar{A}$ 

\begin{equation}
\frac{\bar{A}_{,\bar{\phi}}}{\bar{A}} = \frac{\dot{\bar{A}}}{\bar{A} \dot{\bar{\phi}}} - 
\frac{\bar{A}_{,\bar{X}}}{\bar{A} \dot{\bar{\phi}}} \dot{\bar{X}} -  \frac{\bar{A}_{,\bar{Y}}}{\bar{A}\dot{\bar{\phi}}} \dot{\bar{Y}} -  \frac{\bar{A}_{,\bar{Z}}}{\bar{A}\dot{\bar{\phi}}} \dot{\bar{Z}}
\label{eq:label}
\end{equation}

On the other hand, we now turn to the variations of $X,Y,Z$. First, for $X$ given by \cref{eq:X-expand}. 

\begin{equation}
X = \frac{1}{2}\dot{\bar{\phi}}^{2}+\dot{\bar{\phi}}\,\delta\dot{\phi}-\varphi\,\dot{\bar{\phi}}^{2},
\label{eq:label}
\end{equation}

Taking the time derivative we have 

\begin{equation}
\dot{X} = \dot{\bar{\phi}} \ddot{\bar{\phi}} + \ddot{\bar{\phi}} \delta\dot{\phi} + \dot{\bar{\phi}} \delta \ddot{\bar{\phi}}  -\dot{\varphi}\dot{\bar{\phi}}^2 - 2\varphi \dot{\bar{\phi}} \ddot{\bar{\phi}} 
\label{eq:label}
\end{equation}

Following $\dot{X} = \dot{\bar{X}} + \delta \dot{X}$, we find that the first term is a pure background term, whereas, the 
remaining four terms have first order perturbations. Loosely speaking, we "match the perturbations", thereby retrieveing 

\begin{equation}
\dot{\bar{X}} = \dot{\bar{\phi}} \ddot{\bar{\phi}}, \quad \delta \dot{X} = \ddot{\bar{\phi}} \delta\dot{\phi} + \dot{\bar{\phi}} \delta \ddot{\phi} -\dot{\varphi}\dot{\bar{\phi}}^2 - 2\varphi \dot{\bar{\phi}} \ddot{\bar{\phi}} 
\label{eq:Xbardot}
\end{equation}

For $Y$, since $Y =  \nabla ^ { \mu } \phi \nabla _ { \mu } X$ clearly depends on both $\phi$ and $X$, and in unitary gauge, both of 
these depend only on time, $Y$ must also inherit time dependence. Tracing the steps down up to linear order we have 

\begin{equation}
\begin{aligned}
Y &= \nabla ^ { \mu } \phi \nabla _ { \mu } \\ 
  &= g^{00} \partial_{t} \phi \partial_{t}X \\ 
  &= -(1-2\varphi)(\dot{ \bar{ \phi}} + \delta \dot{\phi})(\dot{\bar{X}} + \delta \dot{X}) \\
  &\approx -(\dot{\bar{\phi}} + \delta \dot{\phi} - 2\varphi \dot{\bar{\phi}})(\dot{\bar{X}} + \delta \dot{X}) \\
  &\approx -\dot{\bar{\phi}}\dot{\bar{X}} + 2\varphi \dot{\bar{\phi}}\dot{\bar{X}} -\dot{\bar{\phi}}\delta \dot{X} - \dot{\bar{X}}\delta \dot{\phi}
\end{aligned}
\label{eq:label}
\end{equation}

Using the same logic as $X$, our background perturbation terms are $\bar{Y} = -\dot{\bar{\phi}}\dot{\bar{X}}$
and $\delta Y = 2\varphi \dot{\bar{\phi}}\dot{\bar{X}} -\dot{\bar{\phi}}\delta \dot{X} - \dot{\bar{X}}\delta \dot{\phi}$. 
Inserting our results from \cref{eq:Xbardot}, we have 

\begin{equation}
\begin{aligned}
Y &= - \dot{\bar{\phi}}^2 \ddot{\bar{\phi}} + 2\varphi \dot{\bar{\phi}}^2 \ddot{\phi} - \dot{\bar{\phi}}\ddot{\bar{\phi}}\delta\dot{\phi} 
- \dot{\bar{\phi}}^2 \delta \ddot{\phi} + \dot{\varphi} \dot{\bar{\phi}}^3 + 2 \varphi \dot{\bar{\phi}}^2 \ddot{\phi} 
- \dot{\bar{\phi}}\ddot{\bar{\phi}}\delta\dot{\phi} \\ 
& = \underbrace{- \dot{\bar{\phi}}^2 \ddot{\bar{\phi}}}_{\bar{Y}} + \underbrace{\bigl ( 4 \dot { \bar { \phi } } ^ { 2 } \ddot { \bar { \phi } } \varphi + \dot { \bar { \phi } } ^ { 3 } \dot { \varphi } - 2 \dot { \bar { \phi } } \ddot { \bar { \phi } } \, \delta \dot { \phi } - \dot { \bar { \phi } } ^ { 2 } \, \delta \ddot { \phi } \bigr )}_{\delta Y}
\end{aligned}
\label{eq:Yexpand}
\end{equation}

It also follows that $Z$ should inherit the time dependence of $X$ and must it must also reduce to partial derivatives 
in unitary gauge 

\begin{equation}
\begin{aligned}
Z &= \nabla^\mu X \nabla_{\mu}X \\ 
  &= g^{\mu \nu} \nabla_{\nu} X\nabla_{\mu} X \\ 
  &= g^{00} \partial_{t} X \partial_{t} X \\ 
  &= -(1-2 \varphi) \dot{X}^2  \\ 
  &= -(1-2 \varphi) (\dot{\bar{X}} + \delta\dot{X})^2 \\  
  &= -(1-2 \varphi)\Big( \dot{\bar{X}}^2 + 2 \dot{\bar{X}}\delta\dot{X} + (\delta\dot{X})^2\Big) \\  
  &\approx -\dot{\bar{X}}^2 - 2\dot{\bar{X}}\delta\dot{X} + 2\varphi\dot{\bar{X}}^2 
     \;\; \cancelto{\mathcal{O}(2)}{\,-4\varphi \dot{\bar{X}} \delta \dot{X}\,}
     \;\; \cancelto{\mathcal{O}(2)}{\,-(\delta\dot{X})^2\,} \\[4pt]
  &\approx - \dot{\bar{\phi}}^2 \ddot{\bar{\phi}}^2
  - 2\dot{\bar{\phi}}\ddot{\bar{\phi}}
    \Big(\ddot{\bar{\phi}} \,\delta\dot{\phi}
        + \dot{\bar{\phi}} \,\delta \ddot{\phi}
        -\dot{\varphi}\,\dot{\bar{\phi}}^2
        - 2\varphi \dot{\bar{\phi}} \ddot{\bar{\phi}}\Big)
  + 2 \varphi \dot{\bar{\phi}}^2 \ddot{\bar{\phi}}^2 \\[4pt]
  &\approx \underbrace{- \dot { \bar { \phi } } ^ { 2 } \ddot { \bar { \phi } } ^ { 2 }}_{\bar{Z}} 
  + \underbrace{\Big( 6 \dot { \bar { \phi } } ^ { 2 } \ddot { \bar { \phi } } ^ { 2 } \varphi
  + 2 \dot { \bar { \phi } } ^ { 3 } \ddot { \bar { \phi } } \dot { \varphi }
  - 2 \dot { \bar { \phi } } \ddot { \bar { \phi } } ^ { 2 } \, \delta \dot { \phi }
  - 2 \dot { \bar { \phi } } ^ { 2 } \ddot { \bar { \phi } } \, \delta \ddot { \phi } \Big)}_{\delta Z}\,.
\end{aligned}
\label{eq:label}
\end{equation}

Now that we have the background values of $X, Y, Z$, by easily taking the time-derivatives thereof we have 

\begin{equation}
\dot{\bar{X}} = \dot{\bar{\phi}}\ddot{\bar{\phi}}, \quad 
\dot{\bar{Y}} = -2\dot{\bar{\phi}}\ddot{\bar{\phi}}^2 - \dot{\bar{\phi}}^2\dddot{\bar{\phi}}, \quad 
\dot{\bar{Z}} = -2\dot{\bar{\phi}} \ddot{\bar{\phi}}^3 - 2\dot{\bar{\phi}}^2 \ddot{\bar{\phi}}\dddot{\bar{\phi}}
\label{eq:label}
\end{equation}

Consequently, \cref{eq:Abarphibar} takes the form 
\begin{equation}
\begin{aligned}
\frac{\bar{A}_{,\bar{\phi}}}{\bar{A}} &= \frac{\dot{\bar{A}}}{\bar{A} \dot{\bar{\phi}}} - 
\frac{\bar{A}_{,\bar{X}}}{\bar{A} \dot{\bar{\phi}}} \dot{\bar{X}} -  \frac{\bar{A}_{,\bar{Y}}}{\bar{A}\dot{\bar{\phi}}} \dot{\bar{Y}} -  \frac{\bar{A}_{,\bar{Z}}}{\bar{A}\dot{\bar{\phi}}} \dot{\bar{Z}} \\ 
&= \frac{\dot{\bar{A}}}{\bar{A} \dot{\bar{\phi}}} 
- \frac{\bar{A}_{,\bar{X}}}{\bar{A} \dot{\bar{\phi}}} \dot{\bar{\phi}}\ddot{\bar{\phi}}
- \frac{\bar{A}_{,\bar{Y}}}{\bar{A}\dot{\bar{\phi}}} \left( -2\dot{\bar{\phi}}\ddot{\bar{\phi}}^2 - \dot{\bar{\phi}}^2\dddot{\bar{\phi}} \right)
- \frac{\bar{A}_{,\bar{Z}}}{\bar{A}\dot{\bar{\phi}}} \left( -2\dot{\bar{\phi}} \ddot{\bar{\phi}}^3 - 2\dot{\bar{\phi}}^2 \ddot{\bar{\phi}}\dddot{\bar{\phi}} \right) \\ 
&= \frac{\dot{\bar{A}}}{\bar{A} \dot{\bar{\phi}}} 
- \frac{\bar{A}_{,\bar{X}}}{\bar{A}}\ddot{\bar{\phi}}
+ \frac{\bar{A}_{,\bar{Y}}}{\bar{A}} \left(2\ddot{\bar{\phi}}^2 + \dot{\bar{\phi}}\dddot{\bar{\phi}} \right)
+ 2\frac{\bar{A}_{,\bar{Z}}}{\bar{A}} \ddot{\bar{\phi}}\left( \ddot{\bar{\phi}}^2 + \dot{\bar{\phi}}\dddot{\bar{\phi}} \right) \\ 
\end{aligned}
\label{eq:label}
\end{equation}  


\bibliographystyle{JHEP}
\bibliography{cosmological-perturbations}


\end{document}